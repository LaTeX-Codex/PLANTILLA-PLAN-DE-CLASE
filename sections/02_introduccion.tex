\section{Introducción}

Para iniciar, el docente se presentará brevemente, compartiendo su experiencia y el propósito de la sesión. Posteriormente, se realizará una evaluación diagnóstica para identificar los conocimientos previos de los estudiantes y ajustar el enfoque de la clase según sea necesario.

A continuación, se presentarán los contenidos que se desarrollarán durante la sesión, destacando su relevancia y cómo estos se vinculan al desarrollo académico y profesional de los estudiantes. Asimismo, se expondrán los objetivos específicos de la clase, los cuales guiarán el aprendizaje y las actividades a realizar.

A lo largo de la clase, se fomentará la participación activa y el análisis crítico para enriquecer el aprendizaje colectivo, asegurando que los contenidos contribuyan al crecimiento integral de los estudiantes.

\subsection*{Preámbulo (5 minutos)}
\begin{itemize}
    \item Aseguramiento del orden, puntualidad, limpieza y disciplina en el aula.
    \item Breve bienvenida a los estudiantes.
    \item Toma de asistencia y verificación de la lista de participantes.
\end{itemize}

\subsection*{Introducción (10 minutos)}
\begin{itemize}
    \item Presentación del contenido de la clase.
    \item Presentación de los objetivos de aprendizaje y estructura de la sesión.
    \item Anuncio de actividades y herramientas a utilizar.
    \item Contextualización de la importancia del contenido de la sesión para el desarrollo académico y profesional.
    \item Pautas de participación y colaboración.
\end{itemize}

\subsection*{Evaluación diagnóstica (15 minutos)}
\begin{itemize}
    \item Evaluación inicial para identificar el nivel de conocimientos previos de los estudiantes.
    \item Uso de herramientas digitales (ej: Kahoot, Google Forms).
    \item Discusión grupal para compartir respuestas y contrastar opiniones.
    \item Retroalimentación inmediata para aclarar dudas y reforzar aprendizajes.
\end{itemize}