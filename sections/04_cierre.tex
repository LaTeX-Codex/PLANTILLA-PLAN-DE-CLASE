\section{Cierre}

\subsection*{Resumen (15 minutos)}
\begin{itemize}
    \item Síntesis de los puntos clave del contenido de la sesión con participación de los estudiantes.
    \item Debate sobre los temas tratados.
    \item Preguntas y respuestas.
\end{itemize}

\subsection*{Verificación de objetivos (10 minutos)}
\begin{itemize}
    \item Dominio de los conceptos clave: \textit{¿Qué aprendieron hoy?}
    \item Aplicación de los conceptos en situaciones prácticas: \textit{¿Cómo pueden aplicar lo aprendido en su vida académica y profesional?}
    \item Resolución de problemas y toma de decisiones: \textit{¿Cómo pueden resolver problemas similares en el futuro?}
    \item Evaluación de habilidades y competencias adquiridas: \textit{¿Qué habilidades desarrollaron durante la sesión?}
    \item Reflexión sobre la relación entre los contenidos y la realidad: \textit{¿Cómo se relacionan los temas tratados con su entorno?}
    \item Reflexión sobre el aprendizaje: \textit{¿Qué fue lo más relevante de la sesión?}
    \item Retroalimentación sobre la metodología y dinámica de la clase: \textit{¿Qué les pareció la forma en que se desarrolló la sesión?}
\end{itemize}

\subsection*{Asignación de trabajo independiente (5 minutos)}
\begin{itemize}
    \item Lectura obligatoria: capítulo de libro, artículo académico, informe técnico.
    \item Investigación individual: búsqueda de información sobre un tema específico o análisis de caso, presentado en un informe, ensayo, exposición o entrada de blog.
    \item Ejercicios prácticos: aplicación de conocimientos mediante la resolución de ejercicios, problemas o casos prácticos.
    \item Práctica de laboratorio: diseño, aplicación, resolución de problemas e implementación de tecnologías, experimentos o simulaciones.
\end{itemize}