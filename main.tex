\documentclass[letterpaper, 12pt, titlepage]{article} % Tipo de documento

% ================================================================
% PAQUETES BÁSICOS Y CONFIGURACIÓN ESENCIAL
% ================================================================
\usepackage[utf8]{inputenc}        % Codificación de caracteres UTF-8
\usepackage[spanish]{babel}        % Localización en español (guiones, títulos)
\usepackage[margin=1in]{geometry}  % Configuración de márgenes de página

% ================================================================
% TIPOGRAFÍA Y FORMATO DE TEXTO
% ================================================================
\usepackage{helvet}                % Fuente Sans-Serif (Helvetica-like)
\renewcommand{\familydefault}{\sfdefault} % Establece Helvet como fuente principal
\usepackage{setspace}              % Control de interlineado (1.5, doble espacio)
\usepackage{parskip}               % Espaciado entre párrafos (en lugar de sangría)
\usepackage{ragged2e}              % Mejor manejo de justificación de texto

% ================================================================
% ELEMENTOS GRÁFICOS Y DISEÑO
% ================================================================
\usepackage{graphicx}              % Manejo de imágenes (inserción y escalado)
\usepackage{xcolor}                % Sistema de colores personalizados
\usepackage{adjustbox}             % Ajuste automático de tablas/objetos
\usepackage{pdflscape}             % Páginas en orientación horizontal

% ================================================================
% TABLAS PROFESIONALES
% ================================================================
\usepackage{booktabs}              % Reglas tipográficas para tablas (formato APA)
\usepackage{xltabular}             % Tablas largas que ocupan todo el ancho del texto
% ================================================================
% LISTAS Y ESTRUCTURAS
% ================================================================
\usepackage{enumitem}              % Personalización avanzada de listas
\setlist{nosep, leftmargin=*, wide=0pt} % Listas ultra compactas
\usepackage{titlesec}              % Formato personalizado para títulos/secciones

% ================================================================
% HIPERVÍNCULOS Y METADATOS PDF
% ================================================================
\usepackage{hyperref}              % Enlaces interactivos y metadatos PDF
\hypersetup{
    colorlinks = true,             % Colores en lugar de cajas
    linkcolor  = blue!70,          % Color para enlaces internos
    urlcolor   = cyan!80,          % Color para URLs
    pdftitle   = {Plan de clase de Asignatura}, % Título del PDF
    pdfauthor  = {Nombre del autor, Nivel educativo}, % Autor del PDF
}

% ================================================================
% ENCABEZADOS Y PIES DE PÁGINA
% ================================================================
\usepackage{fancyhdr}              % Diseño profesional de cabeceras/pies

% ================================================================
% CONFIGURACIONES FINALES
% ================================================================
\renewcommand{\thesection}{\Roman{section}} % Numeración romana para secciones
\newcommand{\universidad}{Universidad de las Regiones Autónomas de la Costa Caribe Nicaragüense} % Nombre de la universidad
\newcommand{\asignatura}{Nombre de la asignatura} % Nombre de la asignatura
\newcommand*{\thead}[1]{\multicolumn{1}{>{\centering\arraybackslash}X}{\bfseries #1}} % Encabezado centrado y en negritas en tablas

\begin{document}

\begin{titlepage}
    \centering
    \begin{minipage}{0.25\textwidth}
        \centering
        \includegraphics[width=0.6\textwidth]{images/uraccan_logo.png}
    \end{minipage}%
    \begin{minipage}{0.75\textwidth}
        \centering
        \Large{\textbf{\universidad}}\\
    \end{minipage}

    \vfill
    \large{\textbf{Plan de clase}}\\
    \Huge{\textbf{\asignatura}}\\
    \vfill
\end{titlepage}
\section{Datos Generales}

\begin{xltabular}{\linewidth}{@{}>{\bfseries}X X@{}}
    \toprule
    Número de clase:                           & Sesión XX de XX               \\
    \midrule
    Fecha:                                     & 03 de marzo de 2025           \\
    \midrule
    Carrera:                                   & Nombre de la carrera          \\
    \midrule
    Asignatura:                                & Nombre de la asignatura       \\
    \midrule
    Año académico:                             & XX año                        \\
    \midrule
    Duración:                                  & 4 horas                       \\
    \midrule
    Turno:                                     & Matutino/Vespertino/Nocturno  \\
    \midrule
    Unidad a desarrollar:                      & Unidad X: Nombre de la unidad \\
    \midrule
    Tema(s):                                   & \begin{itemize}
        \item Tema 1
        \item Tema 2
        \item Tema 3
        \item Tema 4
    \end{itemize}           \\
    \midrule
    Subtema(s):                                & \begin{itemize}
        \item Subtema 1.1
        \item Subtema 1.2
        \item Subtema 1.3
        \item Subtema 1.4
    \end{itemize}        \\
    \midrule
    Objetivos:                                 & \begin{itemize}
        \item Objetivo 1
        \item Objetivo 2
        \item Objetivo 3
        \item Objetivo 4
        \item Objetivo 5
    \end{itemize}               \\
    \midrule
    Formas de Organización de Enseñanzas (FOE): &                               \\
    \midrule
    Medios de enseñanza:                       &                               \\
    \midrule
    Bibliografía:                              & \begin{itemize}
        \item Autor 1, Título 1, Editorial 1, Año 1.
        \item Autor 2, Título 2, Editorial 2, Año 2.
        \item Autor 3, Título 3, Editorial 3, Año 3.
        \item Autor 4, Título 4, Editorial 4, Año 4.
    \end{itemize}                               \\                                         \\
    \bottomrule
\end{xltabular}
\pagebreak

\end{document}